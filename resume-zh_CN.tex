% !TEX TS-program = xelatex
% !TEX encoding = UTF-8 Unicode
% !Mode:: "TeX:UTF-8"

\documentclass{resume}
\usepackage{zh_CN-Adobefonts_external} % Simplified Chinese Support using external fonts (./fonts/zh_CN-Adobe/)
% \usepackage{NotoSansSC_external}
% \usepackage{NotoSerifCJKsc_external}
% \usepackage{zh_CN-Adobefonts_internal} % Simplified Chinese Support using system fonts
\usepackage{linespacing_fix} % disable extra space before next section
\usepackage{cite}
\usepackage{hyperref}

\begin{document}
\pagenumbering{gobble} % suppress displaying page number

\name{张志强}

\basicInfo{
  \email{zhangzhiqiangcs@outlook.com} \textperiodcentered\
  \phone{(+86) 131-2120-0924} \textperiodcentered\ 
  \github[zhangzhiqiangcs]{https://github.com/zhangzhiqiangcs}
} 
\section{教育经历}
\datedsubsection{\textbf{南开大学} \quad 软件学院/软件工程,本科}{2012 -- 2016}
\datedsubsection{\textbf{中科院计算所} \quad 计算机体系结构/区块链,硕士}{2017 -- 2020}

\section{工作经历}
\datedsubsection{\textbf{旷视,北京} \quad 软件工程师}{2020年6月 -- 至今}


\textbf{AI PaaS 平台}
\begin{itemize}[parsep=0.5ex]
  \item 参与 AI PaaS 平台的研发,构建了一个集成\textbf{算力、算法、智能体}管理的统一平台,实现推理系统的标准化和扩展性。
  \item 扩展算法管理模块,支持多种\textbf{大模型}的托管和推理,基于 vLLM 优化推理流程,通过 AI PaaS Gateway 提供 OpenAI API 代理,实现高效的模型推理服务。
  \item 支持多种智能体的接入,提供统一的 API 服务,实现会话管理、历史消息管理功能,提升对话质量和并发能力。
  \item 适配 Kubernetes 运行环境,实现模型推理任务的弹性调度和负载均衡,提高系统的稳定性和资源利用率。
\end{itemize}

\textbf{私有云平台 MDP}
\begin{itemize}[parsep=0.5ex]
  \item 基于 Kubekey 设计并实现了一套标准化 Kubernetes 交付方案,简化了异构环境下的集群部署流程,使 Kubernetes 能够在各种环境中快速落地,提高了系统的可维护性和交付效率。
  \item 构建 Kubernetes 交付平台,集成镜像中心、控制面 HA、监控告警、应用回滚等关键能力。基于 Helm Chart 实现模块化交付,使产品可自由组合组件,简化现场交付流程,提高可复用性和灵活性。
  \item 提供具有最佳实践的\textbf{基础设施能力},涵盖基础设施的可观测性、备份恢复、扩缩容等能力。助力业务快速落地私有云架构。
\end{itemize}

\textbf{私有云平台 DevOps}
\begin{itemize}[parsep=0.5ex]
  \item 基于 Docker, Supervisor 自研容器编排引擎、集群自动化运维核心功能。在 ToB 领域、私有化、重交付场景中实现标准化交付、自动化运维和最佳实践。
  \item \textbf{包管理器}:自研内部包管理器,包含\textbf{命令行工具},\textbf{统一制品仓库},\textbf{控制器}。命令行工具提供依赖解析、配置传递、模板渲染、提交资源的功能。统一制品仓库提供存储、检索、版本管理的能力。控制器监听 apiserver 的资源,调度并部署到 DevOps 平台上。
  \item \textbf{加密授权体系}: 基于 CodeMeter 和 Sentinel 设计并实现加密授权系统,增强核心产品的安全防护能力,防止未授权使用,并支持灵活的授权管理机制。
\end{itemize}

\datedsubsection{\textbf{腾讯,容器云开发组,实习生,北京}}{2019年6月 -- 2019年8月}
\begin{itemize}[parsep=0.5ex]
  \item 参与 腾讯云容器存储(Ceph 集群)的开发和维护,优化存储服务的可靠性和性能。
  \item 排查线上 Ceph 集群异常问题,通过日志分析和 ceph status 监控诊断问题,修复存储节点异常问题,提高系统稳定性。
\end{itemize}


\section{技能}
% increase linespacing [parsep=0.5ex]
\begin{itemize}[parsep=0.5ex]
  \item 精通 Golang,熟悉 Python,具备大型分布式系统开发经验。
  \item 开发平台主要为 Linux,擅长性能调优(profiling、GC 调优),有丰富的线上故障排查经验。
  \item 深入研究 AI Infra(vLLM、TensorRT),云原生(K8s、Operator、Service Mesh),内核技术(eBPF、cgroup)。
  \item 持有 Kubernetes CKA 认证,具备 Kubernetes Operator 开发经验,能独立编写 CRD、Controller。
\end{itemize}

\end{document}
