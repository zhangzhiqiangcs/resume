% !TEX TS-program = xelatex
% !TEX encoding = UTF-8 Unicode
% !Mode:: "TeX:UTF-8"

\documentclass{resume}
\usepackage{zh_CN-Adobefonts_external} % Simplified Chinese Support using external fonts (./fonts/zh_CN-Adobe/)
% \usepackage{NotoSansSC_external}
% \usepackage{NotoSerifCJKsc_external}
% \usepackage{zh_CN-Adobefonts_internal} % Simplified Chinese Support using system fonts
\usepackage{linespacing_fix} % disable extra space before next section
\usepackage{cite}
\usepackage{hyperref}

\begin{document}
\pagenumbering{gobble} % suppress displaying page number

\name{张志强}

\basicInfo{
  \email{zhangzhiqiangcs@outlook.com} \textperiodcentered\
  \phone{(+86) 131-2120-0924} \textperiodcentered\ 
  \github[zhangzhiqiangcs]{https://github.com/zhangzhiqiangcs}
} 
\section{教育经历}
\datedsubsection{\textbf{南开大学} \quad 软件学院/软件工程,本科}{2012 -- 2016}
\datedsubsection{\textbf{中科院计算所} \quad 计算机体系结构/区块链,硕士}{2017 -- 2020}

\section{工作经历}
\datedsubsection{\textbf{旷视,北京} \quad 软件工程师}{2020年6月 -- 至今}
私有云平台1.0(DevOps)
\begin{itemize}[parsep=0.5ex]
  \item \textbf{私有云平台 DevOps}: 基于 Docker, Supervisor 自研容器编排引擎、集群自动化运维核心功能。在 ToB 领域、私有化、重交付场景中实现标准化交付、自动化运维和最佳实践。是旷视所有私有化项目的基础要素。
  \item \textbf{交付工具链}: 自定义 OS 发行版,解决 OS 依赖一致性问题;软件包托管,为交付型软件提供授信存储和分发,提高交付效率。
  \item \textbf{加密授权体系}: 基于 CodeMeter 和 Sentinel 商业产品,构建加密授权平台,为核心产品的安全运行保驾护航。
\end{itemize}

私有云平台2.0(MDP)
\begin{itemize}[parsep=0.5ex]
  \item 借助云原生技术带来的标准化交付和解耦能力,解决产品在私有化交付时的异构环境适配、部署复杂、持续运维问题。
  \item 以业务产品为中心,将 MDP 分别为\textbf{在线平台(Online)}和\textbf{本地平台(Local)}。覆盖产品从研发,测试,交付全流程,提供端到端的工具平台,提升交付体验和交付运维能力。
  \item Online 是面向研发、产品以及交付的平台,提供产品\textbf{编排、发布、部署包管理}等功能
  \item Local 是面向现场部署和运维的管理控制台,提供\textbf{环境搭建、集群管理、应用程序管理}等功能。
  \item 提供具有最佳实践的\textbf{基础设施能力},涵盖基础设施的可观测性、备份恢复、扩缩容等能力。
\end{itemize}

\datedsubsection{\textbf{腾讯,实习生,北京}}{2019年6月 -- 2019年8月}
参与容器云开发组 Ceph 存储服务的开发和维护。
\begin{itemize}[parsep=0.5ex]
  \item 熟悉 Ceph 集群的部署和维护,了解其架构和原理。
  \item 排查线上集群异常问题并提供解决方案。
  \item 调研 Ceph 社区发布的新型存储后端 \href{https://ceph.io/en/news/blog/2017/new-luminous-bluestore/}{bluestore}, 并在内部环境中进行测试。
\end{itemize}


\section{技能}
% increase linespacing [parsep=0.5ex]
\begin{itemize}[parsep=0.5ex]
  \item 工作语言主要为 Golang, 熟悉 Python, Shell 脚本语言, 能够使用 Golang+JS+React 进行前后端开发
  \item 开发平台主要为 Linux, 对服务器端的开发运维、性能调优、故障排查有一定经验
  \item 持续关注云原生, 容器, eBPF等相关技术
\end{itemize}

\end{document}
